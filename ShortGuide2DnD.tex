\documentclass[a4paper]{article}

\usepackage[utf8]{inputenc}
\usepackage[T1]{fontenc}

\begin{document}
\section*{Dungeons \& Dragons - Was ist das?}
Dungeons \& Dragons (kurz D\&D) is ein Pen and Paper Rollenspiel. Das
bedeutet, dass es nur mit Papier und Stift und ein paar Würfeln gespielt
werden kann. Man schlüpft bei D\&D in die Rolle eines Charakters, der in einer
mittelalterlichen Fantasy-Welt Abenteuer erlebt. Das Abenteur selbst wird
dabei von einem Dungeon Master (DM) entworfen. Der DM spielt alle Monster und
Nichtspielercharaktere, die in der Welt auftreten und überlegt sich die
Rahmenhandlung. Im Prinzip - und das ist das tolle and D\&D im Vergleich mit
Computerspielen - sind die Spieler*innen in ihren Handlungen aber sehr frei.
Ob etwas klappt, wird dann durch Würfeln entschieden. Hier ein kleines
Beispiel:\\
Eine Drachengeborene Kriegerin versucht, über eine Felsspalte zu springen, um
die Goblins auf der anderen Seite zu verfolgen. Die DM überlegt sich dann
einen Schwierigkeitsgrad für dieses Vorhaben. Der Schwierigkeitsgrad hängt
hier vor allem davon ab, wie breit die Felsspalte ist. Nehmen wir mal an, der
Schwierigkeitsgrad wird auf 18 festgelet. Die Person, die die Kriegerin
spielt, muss dann mit einem 20er Würfel würfeln. Auf den Wurf werden dann je
nach Situation noch Boni und Mali angerechnet. In diesem Fall würde z.B. der
Fertigkeitswert der Kriegerin für "Athletik" als Bonus angerechnet werden und
es gäbe einen Malus, falls sie bei dem Sprung eine schwere Rüstung trägt. War
der Wert des Würfelwurfs nach Verrechnung aller Boni und Mali größer als 18,
dann war der Sprung erforlgreich und die Kriegerin befindet sich nun auf der
anderen Seite der Schlucht. Ansonsten fällt sie hinein. Wie viel Schaden sie
dabei erleidet, hängt von der Tiefe der Schlucht ab und wird entsprechend
ausgewürfelt.\\
Ihr könnt eure Charaktere in der Welt alles zu tun versuchenh lassen, was ihr
wollt. Wenn es realistisch betrachtet sehr schwer ist (z.B. der Versuch eine
steinerne Tür mit bloßer Hand einzuschlagen) werdet ihr eben mit großer
Wahrscheinlichkeit scheitern. Wenn es eher einfach ist oder euer Charakter in
der entsprechenden Fähigkeit sehr gut ist (z.B. ein geschichtlich belesener
Magier, der versucht, sich an die Taten eines Zwergenkönigs vor 300 Jahren zu
erinnern), dann habt ihr große Erfolgschancen. Es kommt immer mal wieder vor,
dass Spieler*innen etwas versuchen womit der DM nicht gerechnet hat. In diesem
Fall ist das dann mein Problem und ich muss schnell improvisieren und mit
einen Schwierigkeitsgrad überlegen.

\section*{Charaktererstellung - Wie wird man ein gutes Team?}
Die Charaktererstellung ist, wenn man das zum ersten Mal macht, ein bisschen
koompliziert. Wer sich für die Details interessiert, kann im Spielerhandbuch
in Kapitel 2 die erforderlichn Schritte nachlesen. Um das ganze etwas zu
verinfachen, würden Jakob und ich das meiste für euch übernehmen. Wichtig ist
vor allem, dass die Gruppe am Ende ein gutes Team bildet. Daher solltet ihr
euch auch untereinander grob absprechen. Welche Charaktere ihr gerne spielen
wollt. Das entscheidenste Merkmal eines Charakters ist die Klasse. Die Klasse
ist sowas wie eine Ausbildung oder Berufung, Beispiele sind Magierin, Schurke,
Waldläuferin, Kriegsherr, ... Die Klassen lassen sich in folgende Rollen
kategorisieren:
\begin{itemize}
\item Leader
\begin{itemize}
\item Kann als einzige*r Verbündete heilen und ist daher für das Überleben der
Gruppe sehr wichtig
\item Ist oft gut in zwischenmenschlichen Fertigkeiten (Diplomatie,
Einschätzen, Heilkunde)
\item Kann je nach Wahl Nah- oder Fernkämpfer*in sein
\item Klassen: Kriegsherr, Klerikerin (\textsl{engl.} Warlord, Cleric)
\end{itemize}
\item Striker
\begin{itemize}
\item Verursacht am meisten Schaden
\item Ist oft gut in vielen Fertigkeiten (je nach Wahl z.B. Diebeskunst,
Heimlichkeit und Akrobatik oder Naturkunde und Heilkunde)
\item Kann je nach Wahl Nah- oder Fernkämpfer*in sein
\item Klassen: Schurke, Waldläuferin, Hexenmeisterin (\textsl{engl.} Rogue,
Ranger, Warlock)
\end{itemize}
\item Defender (alias Tank)
\begin{itemize}
\item Kann am meisten Schaden vertragen und ist typischerweise sehr gut
gerüstet.
\item Kann Monster dazu bringern, sie anstelle ihrer Verbündeten anzugreifen
und diese damit schützen
\item Ist oft sehr stark und gut in Athletik und Ausdauer
\item Ist immer im Nahkampf
\item Klassen: Kämpferin, Paladin (letztere können auch ein bisschen heilen)
(\textsl{engl.} Fighter, Paladin)
\end{itemize}
\item Controller
\begin{itemize}
\item Kann als einziger mehreren Monster gleichzeitig Schaden
\item Ist oft sehr gut in Wissensfertigkeiten (Magiewissen, Naturkunde,
Geschichte etc.)
\item Ist immer im Fernkampf und typischerweise ziemlich verletzlich in
unmittelbarer Nähe zu Monstern
\item Klasse: Magier (Wizzard)
\end{itemize}
\end{itemize}

\subsection*{Fantasy-Völker}
Es gibt eine Reihe Fantasy-Völker (Zwerge, Elfen, Drachengeborene ...) zu
denen ein Charakter in D\&D gehören kann. Eine Übersicht dazu findet ihr in
Kapitel 3. Jedes Volk lässt sich mit jeder Klasse kombinieren.

\section*{Auftrag an euch}
Bitte sucht euch eine Klasse und ein Volk aus und gebt Jakob und/oder mir Bescheid. Sprecht
euch dabei am besten so ab, dass ihr einigermaßen die unterschiedliche Rollen
abdeckt. Wenn eine offen bleibt, ist das kein Problem: Jakob spielt selbst
auch mit und hat angeboten, den "Lückenfüller" zu machen. Ihr könnt gerne
soviel ihr wollt von eurem Charakter selbst erstellen. Alles, was euch zu
technisch erscheint und worum ihr euch nicht kümmern wollt, lasst ihr Jakob
und mir.Es wäre schön, wenn ihr euch ein paar persönliche Sachen (Name,
Backstory, Motivation nach Abenteuern zu suchen) selbst überlegt. Das macht ja
irgendwie auch Spaß :)

\textbf{Mit Fragen könnt ihr euch grundsätzlich gerne an Jakob und mich
wenden. Bis bald!}
\end{document}
